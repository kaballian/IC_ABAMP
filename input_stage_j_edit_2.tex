\section{Input stage}

The first part of the amplifier is the input stage, this specific implementation is build around a pair of NPN transistors.
This particular implementation includes a tail current source, more specifcally a current mirror build as a sink.


\subsection{topology}
\subsubsection{differential pair}
Q4 and Q5 is an NPN differential pair, with R3 and R2 as collector loads and R15 and R16 as emitter degeneration resistors.
This create a single-ended output differential amplifier. The emitter resistances are included to negate the temperature dependence 
of BJT's and to make the transistor have a more linear behavior. This however changes how the transistors transconductance is calculated
from $gm = \frac{I_C}{V_T}$ to $gm \approx \frac{1}{R_E}$, if and only if $gm R_E\gg 1$.

\begin{figure}[!htb]
    \centering
    \includegraphics[width=0.5\textwidth]{diff_amp.png}
\end{figure}

\textbf{DC analysis}\\
It is assumed that the BC547 has an beta factor of $\beta = 325$ and we aim for a diff amp gain of roughly $ A_{diff} \approx 42$.
The current sink in bottom tries to draw 4mA of current, thus  $I_{Ecm} = 2mA$ must run in each branch of the Amplifier.
The collector resistance then becomes 
\begin{align}
    V_{Ccm} = V_{CC} - R_{Ccm}\cdot I_{Ecm}\alpha \Rightarrow R_{Ccm} = \frac{V_{CC} - V_{Ccm}}{I_{Ecm}\alpha} = 7.774k\Omega
\end{align} 

\textbf{AC analysis}\\
For AC analysis, the intrinsic emitter resistance is $r_e = \frac{V_T}{2mA} = 12.5\Omega$ and the external emitter resistance is $R_E = 75\Omega$
this yields a total emitter resistance of $r_{etot} = 12.5\Omega + 75\Omega = 87.5\Omega$. Thus changes to temperature only affects the term 
with $V_T$, which is $\approx \pm 2mV/C$

The goal here is to relate the differential input voltage $V_d$ to the output voltage $V_o$, by elimintating the internal transistor current $i_b$,
this will yield the gain. Since the desired gain for this Amplifier is chosen, we can use this to find the emitter resistance, which stated earlier is
$R_E = 75\Omega$.

Since the other input of the differential amplifier is grounded, the differential voltage for Q4 becomes:
\begin{align}
    \frac{v_{In diff}}{2} = i_b r_\pi + i_e R_E\\
    r_\pi = \beta r_e\\
    i_e = (\beta + 1)i_b\\
    \Downarrow\\
    v_d = 2i_b(\beta r_e + (\beta + 1)R_E)
\end{align}

And the common mode voltage then becomes:
\begin{align}
    v_{Out diff} = -\beta \cdot i_{bcm}\cdot R_{Ccm}
\end{align}
These two defines the differential gain:
\begin{align}
    A_{diff} = \frac{v_{Out diff}}{v_{In diff}} = -\frac{\beta R_C}{2(\beta r_e + (\beta + 1) R_E)}
\end{align}

Since the gain has already been chosen in the design stage, as mentioned ealier, $ A_{diff} \approx 42$.
The emitter resistance can be calculated, by solving for $R_E$:
\begin{align}
    A_{diff} \rightarrow R_E \approx 75\Omega
\end{align}

It should be noted that the expression for the differential gain gets rather lengthy when expanded, thus for gains much larger
than $\beta \gg 1$, the 1 term in $(\beta +1)$, can safely be ignored. This factors out $\beta$ in both numerator and denominator:
\begin{align}
    2(\beta r_e + (\beta + 1) R_E) \Rightarrow \approx \beta(r_e + R_E)\\
     A_{diff} \approx - \frac{R_C}{2(r_e + R_E)} \approx -42.726
\end{align}

%skriv noget her jakob!
It is desirable to know the common mode rejections ratio, and in order to find that, the common mode gain is needed. 
The common mode gain is given by the common mode input voltage over the common mode output voltage. 
But in order to express them the value of $R_{EE}$ needs to be known. 
$R_{EE}$ is placed to reduce the stress on the transistor in the current mirror. 
An assumption of the collector emitter voltage being 0.3V is made to ensure that the transistor is operating in the active region. Based on this assumsion the expression for $R_{EE}$ is given by

\begin{align}
    0V - V_{BE} - R_E \cdot I_E - 2 \cdot I_E \cdot R_{EE} - V_{CE} - V_{EE} = 0V \\
    R_{EE} = \frac{ V_{BE} - R_E \cdot I_E - V_{CE} - V_{EE}}{2 \cdot I_E} = 3.838 k \Omega \approx 3.9 k \Omega
\end{align}

Now the common mode input voltage can be expressed as:
\begin{align}
    v_{In CM} = i_b \cdot (\beta + 1) \cdot r_e + i_b \cdot (\beta +1) \cdot R_E + 2 \cdot i_b \cdot (\beta +1) \cdot R_{EE}
\end{align}
The times 2 comes from the fact that know there is a contribution from both branches. The output common mode is given by:
\begin{align}
    v_{Out CM} = - \beta \cdot i_b \cdot R_C
\end{align}
With the two above equations the common mode gain can be expressed as:
 \begin{align}
    A_{CM} = \frac{v_{Out CM}}{v_{In CM}} =\frac{ - \beta \cdot i_b \cdot R_C}{ i_b \cdot ( \beta + 1) \cdot r_e + i_b \cdot ( \beta +1) \cdot R_E + 2 \cdot i_b \cdot ( \beta +1)}
\end{align}
In the above equation $i_b$ can be removed because it is multiplied on both the denominator and numerator. Then by combining common expressions in the denominator then the gain can be expressed as:
  
\begin{align}
    A_{CM} =\frac{ - \beta \cdot i_b \cdot R_C }{( \beta +1) \cdot (r_e + R_E + 2\cdot R_{EE})} = - 0.948
\end{align}

Now that both the differential and common mode gain is known the common mode rejection ration can be calculated. It is given as:
\begin{align}
    CMRR = 20 \cdot \log(\frac{A_{diff}}{A_{CM}}) = 33.078 dB
\end{align}


\subsection{Current Sink}
The pupose of the current mirror is to supply, or in this case "sink" the required current for the differential amplifier.
This means that it should be specified to draw the sum of the currents leaving both branches of the diff amp.
Thus the equations becomes:
\begin{align}
    I_{CQ6} = I_{EEcm} = 4mA\\
\end{align}
Because it is a BC547 again, the same $\beta = 325$ is assumed, this means that current entering the base of Q6.
\begin{align}
    I_{BQ6} = \frac{I_{CQ6}}{\beta} \approx 12 \mu A\\
\end{align}
Since the current mirror is comprised to two NPN transistors, the current drawn should be the sum of the collector 
current $I_{CQ5}$ and 2 base currents.
\begin{align}
    I_{RCQ5} = 2 I_{BQ6} + I_{CQ5} = 4.025mA
\end{align}  
Therefore the resistance to define the reference current becomes:
\begin{align}
    R_{ref} = \frac{V_{CC}-V_{EE}}{I_{RCQ5}} \approx 7.5k\Omega
\end{align} 