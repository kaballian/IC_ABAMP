\section{Voltage Amplification Stage}
Following the differential amplifier is a Voltage amplification stage. The purpose of this circuit is to
gain the signal.\\ 

\begin{figure}[!htb]
    \centering
    \includegraphics[width=0.4\textwidth]{VAS.png}
\end{figure}
The topology is a common-emitter amplifier.R10 and R11 is setting the bias point of the stage

\begin{align}
        V_{bQ3} = V_{SS} + \frac{R_{11}}{R_{10}+R_{11}} (V_{CC} - V_{SS}) \approx 1.22V
\end{align}
The Emitter of Q3 will then sit a diode drop below:
\begin{align}
    V_{Eq3} = V_{bQ3} - 0.7V
\end{align}
And the emitter current becomes:
\begin{align}
    I_{Eq3} = \frac{V_{Eq3} - V_{SS}}{R_9} = 10.4mA
\end{align}
%denne strøm skal lige genover vejes, hvis vi skal have ~10mA til at køre igennem RC, så kan det gøre drive Q3 i mætning
%fordi
%VB = 1.22V
%R9 = 1.5k
%VE = VB-0.7V
%IE = (VE-VSS)/1.5k = ~10.3mA


Aiming for a "symmetric swing" around the quiescent collector voltage, meaning that we place the collector voltage at 
halfway between out supply rail and a lower limits, which is set by out NPN transistor. This lower is critical, cause it ensures
that the transistor does not hit saturation, so $V_{CE}$ at the lowest potential should still keep the transistor in the active region.
We can then set it as:
\begin{align}
    V_{Cq} = \frac{V_{CC + 0.82V}}{2} = 7.92V
\end{align}
the collector resistances then becomes: 
\begin{align}
    R_C = \frac{V_{CC} - V_{Cq}}{I_C} 
\end{align}