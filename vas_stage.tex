\section{Voltage Amplification Stage}
Following the differential amplifier is a Voltage amplification stage. The purpose of this circuit is to
gain the signal.\\ 

\begin{figure}[!htb]
    \centering
    \includegraphics[width=0.4\textwidth]{VAS.png}
\end{figure}
The topology is a common-emitter amplifier.R10 and R11 is setting the bias point of the stage
The design goal of this stage is a gain of 3. And to realize this goal
a few other contraints are chosen to make it easier, such as the emitter current
should be $I_E = 5mA$ and $V_{CE} = 7.5V$, to ensure in the active region.\\

From these contrains the ratio of the emitter and collector resistor
can be determined. Also note that this the current design is for DC, thus
for the AC gain an additional impedance can be coupled in to reduce the emitter
resistance in AC and increase the gain.

\begin{align}
    i_B = \frac{v_{in}}{R_{IN}}
    v_o = -R_C i_b \beta
\end{align}
subtituting
\begin{align}
    v_o = -R_C \frac{v_{in}}{R_{IN}} \beta \Rightarrow -\alpha \frac{R_C}{R_E}
\end{align}
Setting $R_C = 3k\Omega$, means that, at AC, $R_E = 1k\Omega$

To achieve the current at DC, the emitter resistance can be calculated:
\begin{align}
    R_{Edc} = \frac{V_{CC}-R_C I_E \alpha - V{CE}-V_{EE}}{I_E} \approx 1.5k\Omega
\end{align}

To compensate for this in AC, an impedance consisting of a resistance and a 
capacitor is set in parallel with the emitter resistance.

\begin{align}
        V_{bQ3} = V_{SS} + \frac{R_{11}}{R_{10}+R_{11}} (V_{CC} - V_{SS}) \approx 1.22V
\end{align}
The Emitter of Q3 will then sit a diode drop below:
\begin{align}
    V_{Eq3} = V_{bQ3} - 0.7V
\end{align}
And the emitter current becomes:
\begin{align}
    I_{Eq3} = \frac{V_{Eq3} - V_{SS}}{R_9} = 10.4mA
\end{align}
%denne strøm skal lige genover vejes, hvis vi skal have ~10mA til at køre igennem RC, så kan det gøre drive Q3 i mætning
%fordi
%VB = 1.22V
%R9 = 1.5k
%VE = VB-0.7V
%IE = (VE-VSS)/1.5k = ~10.3mA


Aiming for a "symmetric swing" around the quiescent collector voltage, meaning that we place the collector voltage at 
halfway between out supply rail and a lower limits, which is set by out NPN transistor. This lower is critical, cause it ensures
that the transistor does not hit saturation, so $V_{CE}$ at the lowest potential should still keep the transistor in the active region.
We can then set it as:
\begin{align}
    V_{Cq} = \frac{V_{CC + 0.82V}}{2} = 7.92V
\end{align}
the collector resistances then becomes: 
\begin{align}
    R_C = \frac{V_{CC} - V_{Cq}}{I_C} 
\end{align}