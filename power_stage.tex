\section{Output Stage}
The output stage consists mainly of circuitry to handle the final power
amplification of the amplification, the purpose
is to gain the currents and not the voltage signal.
Because this is a Class AB, this means that two output 
transistors are driving each half of the signal cycle, however
to change this from class B to AB, it requires transistor biasing, so that
cross distortion is avoided.


\begin{figure}[!htb]
    \centering
    \includegraphics[width=0.7\textwidth]{output_stage.png}
    \caption{Output stage, everything to the right of the red line, is considered
    part of the output stage.}
\end{figure}


To bias the two output transistors, a Vbe multiplier has been implemented.
Q15 and the two resistors R37/38 generate a controlled voltage, 
which is used to determine the working point of the output follow the expression:
\begin{align}
    V_{bias} \approx V_{beQ15} \cdot (1+\frac{R37}{R38}) \approx 2V_{BE}
\end{align}

This then becomes a scale factor for the transistors base emitter voltage, meaning
that the collector voltage will sit an additional drop above it, of $\approx 1.4V$
Practically this circuit also provides the ability to thermally track or follow
the output transistors, due the possibility to physically mount the circuit thermally 
close to the output.\\

The biasing part of the stage is driven by a current sink similar to the one found in 
the input stage, however this sinks $3mA$ of current.
This means that the voltages at the output stages are.

\begin{align}
    V_{R47} = 3mA \cdot 4.8k\Omega = 14.4V\\
    V_{Bm3} = 0.6V\\
    V_{R42} = 3mA \cdot 4.5k\Omega = 13.5V\\
    V_{Bm4} = -15V + 13.5V = -1.5V
\end{align}

Quiescent current through the output stage can then be expressed as:

\begin{align}
    I_{Q} \approx \frac{V_{bias}-(V_{BE,N}+|V_{BE,P|})}{R43+R44}
\end{align}
This indicates that the output stage is somewhat sensitive to changes in
changes in both transistors $V_{be}$ due to fact that those resistances, R43 + R44 are so small.\\

The two emitter resistors linearize the transfer characteritics 
of the output transistors and limits thermal runaway. The main purpose of those
transistors, is to turn small changes in Vbe into current chnages, therefore
the small signal voltage gain is $A_N \approx 1$.
The output impedance, can be approximated by 
\begin{align}
    R_{out} \approx \frac{r_e}{\beta + 1} + R43 + R44
\end{align}
Where $r_e \approx \frac{V_T}{I_E}$

\subsection{Simulation}
Using the calculated values from the analysis, the circuit was then 
simulated in LTspice. The complete circuit can be seen below. 
\begin{figure}[!htb]
    \centering
    \includegraphics[width=0.7\textwidth]{complete_circuit.png}
    \caption{Complete circuit in LTspice.}
\end{figure}
Using the \textit{.tren} spice directive to test the circuit and checking for behavior.





Grounding both of the inputs and probing the simulation model, resulted in a measurement of
1.468V at the collector of Q4 and having its emitter sit a diode drop below ground, means that
this transistor is in the active region. At Dc the current sink through R27 is around 3.6mA.\\

As for the voltage amplification stage, the collector sits at 0.56V and the emitter at 
-7.76V, this means that this coupling also is in the active region. With an emitter current of roughly
4.8mA\\

In the output stage, the current drawn thought the vbe multiplier is roughly 2.94mA, which is close to the 
desired 3mA. The lower bias point sits at $\approx -474mV$, while
the upper bias point is at $\approx 842mV$, this means that there is 
approximately a multiple of 2 base emitter voltage difference between the two points, which is intended.\\

Lastly the Quiescent current, at DC, running through R42 is 1.92mA. This current is flowing to load, as it seems to
be strongly tied to ground. There is only 30mV at the collector of M4, so the Vbe of M4 is not high enough
to draw the current at this stage.




\begin{figure}[!htb]
    \centering
    \includegraphics[]{SIM_output_I.png}
    \caption{RMS current measurement at the output of the amplification, more specifically
    the current running in the load}
\end{figure}
With a RMS current of $I_{OUT} = 95.832mA$ running in the $16\Omega$ load, the power can be calculated:
\begin{align}
    V_{OUT} = I_{RMS} \cdot R_{LOAD} = 95.832mA \cdot 16\Omega = 1.5328 V_{RMS}\\
    P_{OUT} = I_{RMS} \cdot V_{OUT} = 0.146W
\end{align}
This simulated power is a little lower than intended. 
\clearpage
\subsection{Realization}
The circuit was then built and realized:


\begin{figure}[!htb]
    \centering
    \includegraphics[width=0.7\textwidth]{realization.png}
    \caption{implementation on breadboard}
\end{figure}

\begin{figure}[!htb]
    \centering
    \includegraphics[width=0.5\textwidth]{input_sig.png}
    \caption{The chosen input signal is a pure signe wave with a 20mV amplitude 
    at 10kHz}
\end{figure}
\begin{figure}[!htb]
    \centering
    \includegraphics[width=0.7\textwidth]{measurement_graph.png}
    \caption{Measurements of the amplifier. The orange trace is from
    after the differential amplifier, to show much the signal is gained. The blue trace is
    from the output of the amplifier itself, so somewhat larger. The input signal
    is a 10kHz sinewave with an amplitude of 20mV.}
\end{figure}

