\section{Output Stage}
The output stage consists mainly of circuitry to handle the final power
amplification of the amplification, the purpose
is to gain the currents and not the voltage signal.
Because this is a Class AB, this means that two output 
transistors are driving each half of the signal cycle, however
to change this from class B to AB, it requires transistor biasing, so that
cross distortion is avoided.


\begin{figure}[!htb]
    \centering
    \includegraphics[width=0.7\textwidth]{output_stage.png}
    \caption{Output stage, everything to the right of the red line, is considered
    part of the output stage.}
\end{figure}


To bias the two output transistors, a Vbe multiplier has been implemented.
Q15 and the two resistors R37/38 generate a controlled voltage, 
which is used to determine the working point of the output follow the expression:
\begin{align}
    V_{bias} \approx V_{beQ15} \cdot (1+\frac{R37}{R38}) \approx 2
\end{align}

This then becomes a scale factor for the transistors base emitter voltage, meaning
that the collector voltage will sit an additional drop above it, of $\approx 1.4V$
Practically this circuit also provides the ability to thermally track or follow
the output transistors, due the possibility to physically mount the circuit thermally 
close to the output.\\

The biasing part of the stage is driven by a current sink similar to the one found in 
the input stage, however this sinks $3mA$ of current.
This means that the voltages at the output stages are.

\begin{align}
    V_{R47} = 3mA \cdot 4.8k\Omega = 14.4V
    V_{Bm3} = 0.6V
    V_{R42} = 3mA \cdot 4.5k\Omega = 13.5V
    V_{Bm4} = -15V + 13.5V = -1.5V
\end{align}

Quiescent current through the output stage can then be expressed as:

\begin{align}
    I_{Q} \approx \frac{V_{bias}-(V_{BE,N}+|V_{BE,P|})}{R43+R44}
\end{align}

The two emitter resistors linearize the transfer characteritics 
of the output transistors and limits thermal runaway. The main purpose of those
transistors, is to turn small changes in Vbe into current chnages, therefore
the small signal voltage gain is $A_N \approx 1$.
The output impedance, can be approximated by 
\begin{align}
    R_{out} \approx \frac{r_e}{\beta + 1} + R43 + R44
\end{align}
Where $r_e \approx \frac{V_T}{I_E}$

\subsection{Simulation}
\begin{figure}[!htb]
    \centering
    \includegraphics[]{SIM_output_I.png}
\end{figure}

\clearpage
\subsection{Realization}
The circuit was then built and realized:

\begin{figure}
    \centering
    \includegraphics[width=0.7\textwidth]{measurement_graph.png}
    \caption{Measurements of the amplifier. The orange trace is from
    after the differential amplifier, to show much the signal is gained. The blue trace is
    from the output of the amplifier itself, so somewhat larger. The input signal
    is a 10kHz sinewave with an amplitude of 20mV.}
\end{figure}
