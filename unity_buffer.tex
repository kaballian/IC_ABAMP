\section{Unity gain Buffer}
To prevent loading of the voltage amplification stage
into the output stage, a unity gain buffer is implemented
to increase the perceived impedance,  the output of the voltage
amplification stage sees. This is implemented as a 
common collector amplifier with a gain of $\approx$ 1.

\begin{figure}[!htb]
    \centering
    \includegraphics[]{buffer.png}
    \caption{common collector unity gain buffer}
\end{figure}

To ensure that the buffer has low output impedance, so it
does not affect the circuitry that comes after, a decently
sized standing current is necessary. To drive the stage a ratherl
large emitter current of 28mA is chosen.

\begin{align}
    V_E = V_{SS} + I_E R46 = -1V
    V_B = V_B + 0.7V = -0.3V
    I_B = \frac{I_E}{\beta + 1} \approx 85\mu A
\end{align}
The biasing resistor then becomes:
\begin{align}
    R32 = \frac{V_{SS}-V_B}{I_B} \approx 178k\omega
\end{align}

The collector current then becomes:
\begin{align}
    I_C = \frac{\beta}{\beta + 1} I_E \approx 27.9mA
\end{align}

and to sanity check if the current is possible the voltage 
across the collector and emitter must be above 0.3V

\begin{align}
    V_C = Vcc - I_C R49 = -0.074V
    V_{CE} = V_C - V_E = 0.926V
\end{align}

This shows that the unity gain buffer can drive this configuration.\\
The small signal parameters becomes:
\begin{align}
    r_e \approx \frac{V_T}{I_E} = \frac{25mV}{28mA} = 0.893\Omega
    r_{\pi} = \beta r_e \approx 290\Omega
\end{align}

Output impedance, when looking into the emitter. The sources
becomes AC ground, this means that the output impedance becomes
\begin{align}
    R_{out} \approx r_e || 500\Omega \approx r_e
\end{align}



